% Options for packages loaded elsewhere
\PassOptionsToPackage{unicode}{hyperref}
\PassOptionsToPackage{hyphens}{url}
%
\documentclass[
]{article}
\usepackage{lmodern}
\usepackage{amssymb,amsmath}
\usepackage{ifxetex,ifluatex}
\ifnum 0\ifxetex 1\fi\ifluatex 1\fi=0 % if pdftex
  \usepackage[T1]{fontenc}
  \usepackage[utf8]{inputenc}
  \usepackage{textcomp} % provide euro and other symbols
\else % if luatex or xetex
  \usepackage{unicode-math}
  \defaultfontfeatures{Scale=MatchLowercase}
  \defaultfontfeatures[\rmfamily]{Ligatures=TeX,Scale=1}
\fi
% Use upquote if available, for straight quotes in verbatim environments
\IfFileExists{upquote.sty}{\usepackage{upquote}}{}
\IfFileExists{microtype.sty}{% use microtype if available
  \usepackage[]{microtype}
  \UseMicrotypeSet[protrusion]{basicmath} % disable protrusion for tt fonts
}{}
\makeatletter
\@ifundefined{KOMAClassName}{% if non-KOMA class
  \IfFileExists{parskip.sty}{%
    \usepackage{parskip}
  }{% else
    \setlength{\parindent}{0pt}
    \setlength{\parskip}{6pt plus 2pt minus 1pt}}
}{% if KOMA class
  \KOMAoptions{parskip=half}}
\makeatother
\usepackage{xcolor}
\IfFileExists{xurl.sty}{\usepackage{xurl}}{} % add URL line breaks if available
\IfFileExists{bookmark.sty}{\usepackage{bookmark}}{\usepackage{hyperref}}
\hypersetup{
  pdftitle={Untitled},
  hidelinks,
  pdfcreator={LaTeX via pandoc}}
\urlstyle{same} % disable monospaced font for URLs
\usepackage[margin=1in]{geometry}
\usepackage{longtable,booktabs}
% Correct order of tables after \paragraph or \subparagraph
\usepackage{etoolbox}
\makeatletter
\patchcmd\longtable{\par}{\if@noskipsec\mbox{}\fi\par}{}{}
\makeatother
% Allow footnotes in longtable head/foot
\IfFileExists{footnotehyper.sty}{\usepackage{footnotehyper}}{\usepackage{footnote}}
\makesavenoteenv{longtable}
\usepackage{graphicx}
\makeatletter
\def\maxwidth{\ifdim\Gin@nat@width>\linewidth\linewidth\else\Gin@nat@width\fi}
\def\maxheight{\ifdim\Gin@nat@height>\textheight\textheight\else\Gin@nat@height\fi}
\makeatother
% Scale images if necessary, so that they will not overflow the page
% margins by default, and it is still possible to overwrite the defaults
% using explicit options in \includegraphics[width, height, ...]{}
\setkeys{Gin}{width=\maxwidth,height=\maxheight,keepaspectratio}
% Set default figure placement to htbp
\makeatletter
\def\fps@figure{htbp}
\makeatother
\setlength{\emergencystretch}{3em} % prevent overfull lines
\providecommand{\tightlist}{%
  \setlength{\itemsep}{0pt}\setlength{\parskip}{0pt}}
\setcounter{secnumdepth}{5}

\title{Untitled}
\author{}
\date{\vspace{-2.5em}}

\begin{document}
\maketitle

{
\setcounter{tocdepth}{2}
\tableofcontents
}
\hypertarget{table-including--symbol-escapef-not-set}{%
\section{Table including \%-symbol, escape=F not set}\label{table-including--symbol-escapef-not-set}}

\begin{table}[!h]

\caption{\label{tab:unnamed-chunk-1}Caption}
\centering
\begin{threeparttable}
\begin{tabular}[t]{lcccc}
\toprule
Variable & \textbackslash{}makecell[c]\{All\textbackslash{}\textbackslash{}(n = 300)\} & \textbackslash{}makecell[c]\{Group1\textbackslash{}\textbackslash{}(n = 120)\} & \textbackslash{}makecell[c]\{Group2\textbackslash{}\textbackslash{}(n = 100)\} & \textbackslash{}makecell[c]\{Group3\textbackslash{}\textbackslash{}(n = 80)\}\\
\midrule
Var1 & 31\% & 79\% & 51\% & 14\%\\
Var2 & 67\% & 42\% & 50\% & 43\%\\
Var3 & 14\% & 25\% & 90\% & 91\%\\
Var4 & 69\% & 91\% & 57\% & 92\%\\
\bottomrule
\end{tabular}
\begin{tablenotes}
\item \textit{Anmerkung:} 
\item * This is a note to show what * shows in this table plus some addidtional words to make this string a bit longer. Still a bit more
\end{tablenotes}
\end{threeparttable}
\end{table}

\hypertarget{table-including--symbol-escapef}{%
\section{Table including \%-symbol, escape=F}\label{table-including--symbol-escapef}}

\begin{verbatim}
This leads to an Error, see pic below
\end{verbatim}

\begin{table}[!h]

\caption{\label{tab:unnamed-chunk-2}Caption}
\centering
\begin{threeparttable}
\begin{tabular}[t]{lcccc}
\toprule
Variable & \makecell[c]{All\\(n = 300)} & \makecell[c]{Group1\\(n = 120)} & \makecell[c]{Group2\\(n = 100)} & \makecell[c]{Group3\\(n = 80)}\\
\midrule
Var1 & 31\% & 79\% & 51\% & 14\%\\
Var2 & 67\% & 42\% & 50\% & 43\%\\
Var3 & 14\% & 25\% & 90\% & 91\%\\
Var4 & 69\% & 91\% & 57\% & 92\%\\
\bottomrule
\end{tabular}
\begin{tablenotes}
\item \textit{Anmerkung:} 
\item * This is a note to show what * shows in this table plus some addidtional words to make this string a bit longer. Still a bit more
\end{tablenotes}
\end{threeparttable}
\end{table}

\end{document}
